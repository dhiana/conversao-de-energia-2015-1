\documentclass[conference,harvard,brazil,english]{sbatex}
\usepackage[latin1]{inputenc}
\usepackage{ae}
\makeatletter
\def\verbatim@font{\normalfont\ttfamily\footnotesize}
\makeatother
\usepackage{amsmath}

\begin{document}

\title{Controle de M�quina de Corrente Cont�nua}
\author{Dhiana D. C. Rocha (106077690)}{dhiana.deva@poli.ufrj.br}
\author[1]{Rafael Prallon (108039591)}{rafael.prallon@poli.ufrj.br}
\address{Departamento de Engenharia Eletr�nica e de Computa��o\\
         Escola Polit�cnica\\
         Universidade Federal do Rio de Janeiro}

\twocolumn[

\maketitle

\selectlanguage{english}
\begin{abstract}
This work presents a project for controlling a continuous current machine.
First, the principles of such machines are explained and then the project is detailed.
\end{abstract}
\keywords{Continuous Current Machine, Motor, Generator.}

\selectlanguage{brazil}
\begin{abstract}
Este trabalho apresenta um projeto para controlar uma m�quina em corrente cont�nua.
Primeiramente, os prin�pios de tais m�quinas s�o explicados e em seguida o projeto � detalhado.
\end{abstract}
\keywords{M�quina de Corrente Cont�nua, Motor, Gerador.}

]

\selectlanguage{brazil}

\section{Introdu��o}

M�quinas de Corrente Cont�nua (CC) podem ser utilizadas com dois prop�sitos:

\begin{description}
\item[Motor:] converte energia el�trica em energia mec�nica.
\item[Gerador:] converte energia mec�nica em energia el�trica.
\end{description}

Este trabalho apresenta o funcionamento de tais m�quinas nestes dois modos.
Adicionalmente, � projetado um controlador PID para um motor CC.

\section{Modelo Te�rico}

A for�a eletromotriz � dada por:

\begin{equation}
E_a = k_{armadura} * \omega * \phi
\end{equation}

Sendo o fluxo de campo magn�tico dado por:

\begin{equation}
\phi = k_{campo} * I_f
\end{equation}

Equacionando as malhas do modelo te�rico:

\begin{equation}
V_d = I_a * R_a + L_a * \frac{dI_a}{dt} + E_a
\end{equation}

\begin{equation}
V_d = I_f * R_f + L_f * \frac{dI_f}{dt}
\end{equation}

Chega-se �s correntes:

\begin{equation}
I_a(s) = \frac{V_d - E_a}{L_a * s + R_a}
\end{equation}

\begin{equation}
I_f(s) = \frac{V_f}{L_f * s + R_f}
\end{equation}

O torque el�trico � dado por:

\begin{equation}
T_{eletrico} = k_{armadura} * \phi * I_a
\end{equation}

\begin{equation}
T_{eletrico} = k_{armadura} * k_{campo} * I_f * I_a
\end{equation}

\begin{equation}
T_{eletrico} = k * \frac{V_f}{R_f} * I_a
\end{equation}

E o torque mec�nico � dado por:

\begin{equation}
T_{mecanico} = F \times R
\end{equation}

\begin{equation}
T_{mecanico} = m * a * R
\end{equation}

J� o torque devido ao atrito � dado por:

\begin{equation}
T_{atrito} = -k_a * \omega
\end{equation}

Finalmente, o torque total � composto por:

\begin{equation}
T_{total} = T_{eletrico} + T_{atrito} - T_{mecanico}
\end{equation}

\section{M�quina de Corrente Cont�nua}

\subsection{Estudo do Motor}

Texto

\subsection{Estudo do Gerador}

Texto

\subsection{Din�mica do Motor}

Texto

\subsection{Simula��o do Motor}

Texto

\section{Controlador para o Motor de Corrente Cont�nua}

\subsection{Movimento em Tempo M�nimo}

Texto

\subsection{Movimento com Velocidade Constante}

Texto

\subsection{Movimento com Acelera��o em Rampa}

Texto

\subsection{Raio �timo}

Texto

\subsection{Consumo de energia total}

Texto

\subsection{Sensores utilizados}

Texto

\section{Conclus�es}

Texto

\nocite{*}
\bibliography{exemplo}

\end{document}
