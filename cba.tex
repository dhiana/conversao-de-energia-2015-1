\documentclass[conference,harvard,brazil,english]{sbatex}
\usepackage[latin1]{inputenc}
\usepackage{ae}
\makeatletter
\def\verbatim@font{\normalfont\ttfamily\footnotesize}
\makeatother
\usepackage{amsmath}

\begin{document}

\title{Controle de M�quina de Corrente Cont�nua}
\author{Dhiana D. C. Rocha (106077690)}{dhiana.deva@poli.ufrj.br}
\author{Rafael Prallon (108039591)}{rafael.prallon@poli.ufrj.br}

\twocolumn[

\maketitle

\selectlanguage{english}
\begin{abstract}
Abstract
\end{abstract}
\keywords{Template, Example.}

\selectlanguage{brazil}
\begin{abstract}
Resumo
\end{abstract}
\keywords{Exemplo, Ilustra��o.}

]

\selectlanguage{brazil}

\section{Introdu��o}

Texto

\section{M�quina de Corrente Cont�nua}

\subsection{Estudo do Motor}

Texto

\subsection{Estudo do Gerdor}

Texto

\subsection{Din�mica do Motor}

Texto

\subsection{Simula��o do Motor}

Texto

\section{Controlador para o Motor de Corrente Cont�nua}

\subsection{Movimento em Tempo M�nimo}

Texto

\subsection{Movimento com Velocidade Constante}

Texto

\subsection{Movimento com Acelera��o em Rampa}

Texto

\section{Conclus�es}

Texto

\bibliography{exemplo}

\end{document}
