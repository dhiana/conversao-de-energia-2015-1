\documentclass[conference,harvard,brazil,english]{sbatex}
\usepackage[latin1]{inputenc}
\usepackage{ae}
\makeatletter
\def\verbatim@font{\normalfont\ttfamily\footnotesize}
\makeatother
\usepackage{amsmath}

\begin{document}

\title{Controle de M�quina de Corrente Cont�nua}
\author{Dhiana D. C. Rocha (106077690)}{dhiana.deva@poli.ufrj.br}
\author[1]{Rafael Prallon (108039591)}{rafael.prallon@poli.ufrj.br}
\address{Departamento de Engenharia Eletr�nica e de Computa��o\\
         Escola Polit�cnica\\
         Universidade Federal do Rio de Janeiro}

\twocolumn[

\maketitle

\selectlanguage{english}
\begin{abstract}
This work presents a project for controlling a continuous current machine.
First, the principles of such machines are explained and then the project is detailed.
\end{abstract}
\keywords{Continuous Current Machine, Motor, Generator.}

\selectlanguage{brazil}
\begin{abstract}
Este trabalho apresenta um projeto para controlar uma m�quina em corrente cont�nua.
Primeiramente, os prin�pios de tais m�quinas s�o explicados e em seguida o projeto � detalhado.
\end{abstract}
\keywords{M�quina de Corrente Cont�nua, Motor, Gerador.}

]

\selectlanguage{brazil}

\section{Introdu��o}

M�quinas de Corrente Cont�nua (CC) podem ser utilizadas com dois prop�sitos:

Motor: converte energia el�trica em energia mec�nica.

Gerador: converte energia mec�nica em energia el�trica.

Este trabalho apresenta o funcionamento de tais m�quinas nestes dois modos.
Adicionalmente, um controlador PID ser� projetado para controlar um motor CC.

\section{M�quina de Corrente Cont�nua}

\subsection{Estudo do Motor}

Texto

\subsection{Estudo do Gerador}

Texto

\subsection{Din�mica do Motor}

Texto

\subsection{Simula��o do Motor}

Texto

\section{Controlador para o Motor de Corrente Cont�nua}

\subsection{Movimento em Tempo M�nimo}

Texto

\subsection{Movimento com Velocidade Constante}

Texto

\subsection{Movimento com Acelera��o em Rampa}

Texto

\section{Conclus�es}

Texto

\nocite{*}
\bibliography{exemplo}

\end{document}
